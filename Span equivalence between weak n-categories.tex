\documentclass{tac}

% Prepared using tac.cls and diagxy (if you do not have diagxy, compiling this will require commenting lines 165-168)
% PLEASE READ comments in the text below
% PLEASE NOTE: source files for submission to TAC require a comment like the following
%              giving style, packages used, TeX implementation
% TAC style, 2 pp, Xy-pic ver 3.7, MikTeX version 3.1


% NOTE: packages used should be the first part of the preamble

\usepackage{amsmath}
\usepackage{amssymb}
%\usepackage{amsfonts}

\usepackage[all]{xy}

% diagxy loads xy, so the line preceding is redundant

\input diagxy

\def\xypic{\hbox{\rm\Xy-pic}}

% The TAC hyperref setup should be reloaded after other packages

\usepackage[colorlinks=true]{hyperref}
\hypersetup{allcolors=[rgb]{0.1,0.1,0.4}}


% NOTE: TAC preamble macros come next...

\author{Yuya Nishimura}

% NOTE: that \thanks is outside the \author macro, unlike article style...

\thanks{}

\address{}


\title {Span equivalence between weak $n$-categories}

% NOTE: this is required...

\copyrightyear{2015}

% NOTE: the next three are optional in this style (but are required for TAC publication)

\keywords{TAC, diagrams}
\amsclass{00A00}

% NOTE: that \CR here provides a vertical listing
\eaddress{}


% NOTE: author macros  BEGIN here
%       (they are all actually used in the article!!)
% *PLEASE* note the begin and end of author macros in the source file

\def\xypic{\hbox{\rm\Xy-pic}}

% examples of proclamation macros follow - these define environments like \begin{thm} ... \end{thm}
% by default they are italic, but roman proclamations for remarks, examples are also available
% with \newtheoremrm{}{}

\newtheorem{thm}{Theorem}
% note that \newtheorem{theorem}{Theorem} and several others are already in tac.sty
\newtheorem{theorem}{Theorem} 
\newtheorem{lem}{Lemma}

% note rm This is useful for remarks and examples
\newtheoremrm{rem}{Remark}

%diagrams
\newcommand{\pullbackmark}[2]{\save ;p+<.8pc,0pc>:(0,-1)::%
(#1) *{\phantom{Z}} %
;p+(#2)-(0,0) **@{-}%
;p-(#1)+(0,0) *{\phantom{Z}} **@{-} \restore}

% TAC has a \proof environment, with abbreviations following; note also \mathrmdef{} and \mathbfdef{}

\let\pf\proof
\let\epf\endproof
\mathrmdef{Hom}
\mathbfdef{Set}

% author macros END here

\begin{document}

\maketitle
\begin{abstract}

\end{abstract}

% NOTE: it is good practice to \label all headings (and proclamations) immediately

%NOTE: more verbosely, the next line could use \begin{lem} ... \end{lem}

%\lem \endlem
%\pf 
%\epf


%\thm \endthm
%\pf
%\epf

% NOTE: If you use \cite at the beginning of a theorem-like environment,
% it must come immediately after the theorem and before any label.

\section{Introduction}

\section{Span equivalence}

\begin{definition}
Let $n \in \mathbb{N}$, An $n$-globular set is a diagram 
\[
\xymatrix{
X = ( X_n \ar@<0.5ex>[r]^{s_n^X} \ar@<-0.5ex>[r]_{t_n^X} & X_{n-1} \ar@<0.5ex>[r]^{s_{n-1}^X} \ar@<-0.5ex>[r]_{t_{n-1}^X} 
 & ... \ar@<0.5ex>[r]^{s_1^X} \ar@<-0.5ex>[r]_{t_1^X} & X_{0}  )
 }
\]
of sets and maps such that
\[
s_{k-1}^X s_{k}^X (x) =s_{k-1}^X t_{k}^X (x) , \hspace{10pt} t_{k-1}^X s_{k}^X (x) = t_{k-1}^X t_{k}^X (x)
\]
for all $k \in \{ 2,...,n \} $ and $x \in X_k$. \\
Elements of $X_k$ are called $k$-cells of $X$. And we defined hom-sets of $X$ as follows:
\[
{\bf Hom}(x,y) := \{ \alpha \in X_k \mid s_k^X(\alpha)=x, t_k^X(\alpha)=y \}
\] 
for all $k \in \{ 1,...n \} $ and $x,y \in X_{k-1}$.\\
Let $X,Y$ be $n$-globular sets, A map of $n$-globular sets from $X$ to $Y$ is a collection $f=\{ f_{k}: X_{k} \rightarrow Y_{k} \}_{k \in \{ 1,...,n \}}$ of maps of sets such that 
\[
s_k^Y f_k (x) = f_{k-1} s_{k}^X (x) , \hspace{10pt} t_{k}^Y f_k (x) = f_{k-1} t_{k}^X (x)
\]
for all $k \in \{ 1,...,n \} $ and $x \in X_{k}$. \\
The category of $n$-globular sets and maps is written $n \mathchar`- {\bf GSet}$.
\end{definition}



\begin{definition}
Let $f:X \rightarrow Y$ be a map of $n$-globular sets.
\begin{list}{$\bullet$}{}
\item $f$:surjective on $k$-cells $ : \Leftrightarrow$ $f_k : X_k \rightarrow Y_k $ :surjective
\item $f$:injective on $k$-cells $ : \Leftrightarrow$ $f_k : X_k \rightarrow Y_k $ :injective
\item $f$:full on $k$-cells $ : \Leftrightarrow$ 
$\left\{ \begin{array}{l} 
\forall x, x' \in X_{k-1}, g \in {\bf Hom}_{Y}(f(x), f(x')), \\
\exists h \in {\bf Hom}_{X}(x,x') \hspace{5pt} {\rm s.t.} \hspace{5pt} f(h)=g \\
\end{array} \right.$
\item $f$:faithful on $k$-cell $ : \Leftrightarrow$ 
$\left\{ \begin{array}{l}{}
\forall x,x' \in X_{k-1}, g,g' \in {\bf Hom}_{X}(f(x),f(x')), \\
g \neq g' \Rightarrow f(g) \neq f(g')
\end{array} \right.$
\end{list}
\end{definition}

\begin{definition}
Let $K$ be an $n$-globular operad. $K$-algebras $KX \rightarrow X$ and $KY \rightarrow Y$ are span equivalent if there exists a triple $\langle \psi , u, v \rangle$ such that $\psi : KZ \rightarrow Z$ is an $K$-algebra, $u: Z \rightarrow X$ and $v: Z \rightarrow Y$ are maps of $K$-algebras, surjective on $0$-cells, full on $m$-cells for all $1 \leq m \leq n$, and faithful on $n$-cells.The triple $\langle \psi , u, v \rangle$ is referred to as an span equivalence of $K$-algebras.
\end{definition}

\begin{proposition}
In the pullback diagram in $n \mathchar`- {\bf GSet}$
\[ \xymatrix{
 P \pullbackmark{1,0}{0,1} \ar[r]^j \ar[d]_i & Y \ar[d]^g \\
 X \ar[r]_f & S
} \]
\begin{list}{$\bullet$}{}
\item $f$:surjective on $0$-cells$\Rightarrow$ $j$:surjective on $0$-cells
\item $f$:full on $k$-cells $\Rightarrow$ $j$:full on $k$-cells
\item $f$:faithful on $k$-cells $\Rightarrow$ $j$:faithful on $k$-cells
\end{list}
\end{proposition}

\pf We define $n$-globular set $P$ as follows:
\[ P_k := \{ (x,y) \in X_k \times Y_k \mid f_k(x)=g_k(y) \}  \]
\[ s_l^P := ( P_l \ni (x,y) \mapsto (s_l^X (x), s_l^Y (y) ) \in P_{l-1} ) \]
\[ t_l^P := ( P_l \ni (x,y) \mapsto (t_l^X (x), t_l^Y (y) ) \in P_{l-1} ) \]
for all $k \in \{ 0,...,n \}, l \in \{ 1,...,n \}$, and maps of $n$-globular sets $i,j$ as follows:
\[
i_k := (P_k \ni (x,y) \mapsto x \in X_k), \hspace{10pt} j_k := (P_k \ni (x,y) \mapsto y \in Y_k)
\]
for all $k \in \{ 0,...,n \}$. Then $(P,i,j)$ is a pullback of $X$ and $Y$ over $S$. \\
First, let $f$ be surjective on $0$-cells. For $y \in Y_0$, there exists $x \in X_0$ such that $f_{0}(x)=g_{0}(y)$, Then $(x,y) \in P_{0}$ and $j_{0}((x,y))=y$. Therefore $j$ is surjective on $0$-cells.\\
Second, let $f$ be full on $k$-cells. let $(x,y), (x',y') \in P_{k-1}, \phi \in {\bf Hom}(y,y')$, then $s_{k} g_{k}(\phi)=g_{k-1}(y)=f_{k-1}, t_{k} g_{k} (\phi)=g_{k-1} (y')=f_{k-1} (x')$. Thus $g_{k} (\phi) \in {\bf Hom}(f_{k-1}(x), f_{k-1}(x')$. From fullness, there exists $\psi \in {\bf Hom}(x,x')$ such that $f_k (\psi) = g_k (\phi)$. Then $(\psi , \phi ) \in {\bf Hom}((x,y),(x',y'))$ and $j_k (\psi ,\phi ) =\phi$. Therefore $j$ is full on $k$-cells. \\
Third, let $f$ be faithful on $k$-cells. let $(x,y), (x',y') \in P_{k-1}, \psi, \phi \in {\bf Hom}((x,y), (x', y'))$ such that $j_{k}(\psi)=j_{k}(\phi)$. Then $f_k i_k(\psi)=g_k j_k (\psi)=g_k j_k (\phi)=f_k i_k(\phi)$. From faithfulness, $i_k (\psi)= i_k (\phi)$, and $\psi = (i_k (\psi ), j_k (\psi ))=(i_k (\phi ), j_k (\phi ))=\phi $. Therefore $j$ is faithful on $k$-cells.

 

\begin{rem}
Let $K$ be a monad on $n \mathchar`- {\bf GSet}$. Pullbacks in $K \mathchar`- {\bf Alg}$ are created by the forgetful functor $U: K \mathchar`- {\bf Alg} \rightarrow n \mathchar`- {\bf GSet}$.
\end{rem}

\begin{proposition}
Let
\[ \xymatrix{
 & \ar[ld]_f P \ar[rd]^g &  & & & \ar[ld]_h Q \ar[rd]^i & \\
X & & Y &  & Y & & Z 
} \]
be span equivalences, then
\[ \xymatrix{ 
 & & \ar[ld]_p R \pullbackmark{-1,1}{1,1} \ar[rd]^q & & \\
 & \ar[ld]_f P \ar[rd]^g &  & \ar[ld]_h Q \ar[rd]^i & \\
X & & Y & & Z 
} \]
is span equivalence.
\end{proposition}

\pf  By proposition, $p,q$ are are surjective on $0$-cells, full on $k$-cells for $1 \leq k \leq n$ and faithful on $n$-cells. Therefore $f \circ p , i \circ q$ are surjective on $0$-cells, full on $k$-cells for $1 \leq k \leq n$ and faithful on $n$-cells. So the span is span equivalence.\\

\begin{thm}
Span equivalence is equivalence relation on $K$-algebras.
\end{thm}

\pf It is straightforward from the definition and previous proposition that span equivalence is equivalence relation.


\section{Characterizing equivalence of categories via spans}

\begin{definition}
Let ${\cal A}$ and ${\cal B}$ be categories. We say that ${\cal A}$ and ${\cal B}$ are span equivalent if there exists a triple $\langle {\cal A}, u, v \rangle$ such that ${\cal C}$ is a category, $u: {\cal C} \rightarrow {\cal A}$ and $v: {\cal C} \rightarrow {\cal B}$ are functors, surjective on objects, full and faithful.
\end{definition}

\begin{definition}
Let ${\cal A}$ and ${\cal B}$ be categories, let $\langle S:{\cal A}\rightarrow{\cal B}, T:{\cal B} \rightarrow {\cal A}, \eta :I_{{\cal A}} \rightarrow TS , \epsilon :ST \rightarrow I_{{\cal B}} \rangle$ be an adjoint equivalence between ${\cal A}$ and ${\cal B}$. We define a category, {\rm equivalence fusion} ${\cal A}\sqcup \hspace{-.76em} \mid {\cal B}$ , as follows:

\begin{list}{$\bullet$}{}
\item object-set
\[ {\bf Ob}{\rm (}{\cal A}\sqcup \hspace{-.76em} \mid {\cal B}{\rm )} := {\bf Ob}{\rm (}{\cal A}{\rm )} \bigsqcup {\bf Ob}{\rm (}{\cal B}{\rm )} \hspace{20pt}  {\rm (disjoint)} \]
\item hom-set
\[ {\bf Hom}{\rm (}x,y{\rm )} := \left\{ \begin{array}{ll}
\{ \langle f,x,y \rangle \mid f \in {\cal A}{\rm (}x,y{\rm )} \} & {\rm (}x,y \in {\cal A}{\rm )} \\
\{ \langle f,x,y \rangle \mid f \in {\cal B}{\rm (}x,y{\rm )} \} & {\rm (}x,y \in {\cal B}{\rm )} \\
\{ \langle f,x,y \rangle \mid f \in {\cal B}{\rm (}Sx,y{\rm )} \} & {\rm (}x \in {\cal A}, y \in {\cal B}{\rm )} \\
\{ \langle f,x,y \rangle \mid f \in {\cal B}{\rm (}x,Sy{\rm )} \} & {\rm (}x \in {\cal B}, y \in {\cal A}{\rm )} \\
\end{array} \right. \]
\item composition
\[ \begin{array}{ll}
\tilde{\circ} : {\bf Hom}{\rm (}y,z{\rm )} \times {\bf Hom}{\rm (}x,y{\rm )} & \longrightarrow {\bf Hom}{\rm (}x,z{\rm )} \\
\hspace{50pt} \langle \langle g,y,z \rangle , \langle f,x,y \rangle \rangle & \longmapsto \langle g,y,z \rangle \tilde{\circ} \langle f,x,y \rangle := \langle g \circ f,x,z \rangle 
\end{array} \]
 \[ g \circ f := \left\{ \begin{array}{ll}
 g \circ_{{\cal A}} f & {\rm (}x,y,z \in {\cal A}{\rm )} \\
 g \circ_{{\cal B}} f & {\rm (}x,y,z \in {\cal B}{\rm )} \\
 g \circ_{{\cal B}} Sf & {\rm (}x,y \in {\cal A}, z \in {\cal B}{\rm )} \\
 g \circ_{{\cal B}} f & {\rm (}x \in {\cal A}, y,z \in {\cal B}{\rm )} \\
 g \circ_{{\cal B}} f & {\rm (}x,y \in {\cal B}, z \in {\cal A}{\rm )} \\
 Sg \circ_{{\cal B}} f & {\rm (}x \in {\cal B}, y,z \in {\cal A}{\rm )} \\
 \eta_{z}^{-1} \circ_{{\cal A}} Tg \circ_{{\cal A}} Tf \circ_{{\cal A}} \eta_{x} & {\rm (}x \in {\cal A}, y \in {\cal B}, z \in {\cal A}{\rm )} \\
 g \circ_{{\cal B}} f & {\rm (}x \in {\cal B}, y \in {\cal A}, z \in {\cal B}{\rm )} \\
\end{array} \right. \]
\item identities
\[ {\rm id}_{x} :=\left\{ \begin{array}{ll}
\langle {\rm id}_{x},x,x \rangle & {\rm (}x \in {\cal A}, {\rm id}_{x} \in {\cal A}{\rm (}x,x{\rm )} {\rm )} \\
\langle {\rm id}_{x},x,x \rangle & {\rm (}x \in {\cal B}, {\rm id}_{x} \in {\cal B}{\rm (}x,x{\rm )} {\rm )} \\
\end{array} \right. \]
\end{list}
\end{definition}


\begin{proposition}
The equivalence fusion ${\cal A}\sqcup \hspace{-.76em} \mid {\cal B}$ forms a category.
\end{proposition}

\pf
It is easy to check that the composition $\tilde{\circ}$ is map from ${\bf Hom}(x,y) \times {\bf Hom}(y,z)$ to ${\bf Hom}(x,z)$. Now, we prove that the composition $\tilde{\circ}$ satisfies composition law and identity law by case analysis.
\begin{itemize}

\item composition law
 \begin{itemize}
 \item $x \in {\cal A}, y \in {\cal A}, z \in {\cal A}, w \in {\cal A}$, \\
 $h \circ ( g \circ f ) = h \circ_{{\cal A}} ( g \circ_{{\cal A}} f ) $\\
 $( h \circ g ) \circ f = ( h \circ_{{\cal A}} g ) \circ_{{\cal A}} f $
 \item $x \in {\cal A}, y \in {\cal A}, z \in {\cal A}, w \in {\cal B}$, \\
 $h \circ ( g \circ f ) = h \circ ( g \circ_{{\cal A}} f ) = h \circ_{{\cal B}} S(g \circ_{{\cal A}} f) = h \circ_{{\cal B}} ( Sg \circ_{{\cal B}} Sf )$\\
 $( h \circ g ) \circ f = ( h \circ_{{\cal B}} Sg ) \circ f = ( h \circ_{{\cal B}} Sg ) \circ_{{\cal B}} Sf $
 \item $x \in {\cal A}, y \in {\cal A}, z \in {\cal B}, w \in {\cal A}$, \\
 $h \circ ( g \circ f ) = h \circ ( g \circ_{{\cal B}} Sf ) = \eta_{w}^{-1} \circ_{{\cal A}} Th \circ_{{\cal A}} T(g \circ_{{\cal B}} Sf) \circ_{{\cal A}} \eta_{x}  \\
 \hspace{134pt} =\eta_{w}^{-1} \circ_{{\cal A}} Th \circ_{{\cal A}} Tg \circ_{{\cal A}} TSf \circ_{{\cal A}} \eta_{x} \\
  \hspace{134pt} =\eta_{w}^{-1} \circ_{{\cal A}} Th \circ_{{\cal A}} Tg \circ_{{\cal A}} \eta_{y} \circ_{\cal A} f  \\
 ( h \circ g ) \circ f = ( \eta_{w}^{-1} \circ_{{\cal A}} Th \circ_{{\cal A}} Tg \circ_{{\cal A}} \eta_{y} ) \circ f =( \eta_{w}^{-1} \circ_{{\cal A}} Th \circ_{{\cal A}} Tg \circ_{{\cal A}} \eta_{y} ) \circ_{{\cal A}} f $
 \item $ x \in {\cal A}, y \in {\cal A}, z \in {\cal B}, w \in {\cal B}$, \\
 $h \circ ( g \circ f ) = h \circ ( g \circ_{{\cal B}} Sf ) = h \circ_{{\cal B}} ( g \circ_{{\cal B}} Sf )$\\
$( h \circ g ) \circ f = ( h \circ_{{\cal B}} g ) \circ f =  ( h \circ_{{\cal B}} g ) \circ_{{\cal B}} Sf $
 \item $x \in {\cal A}, y \in {\cal B}, z \in {\cal A}, w \in {\cal A}$, \\
 $h \circ ( g \circ f ) = h \circ ( \eta_{z}^{-1} \circ_{{\cal A}} Tg  \circ_{{\cal A}}  Tf  \circ_{{\cal A}}  \eta_{x}) = h  \circ_{{\cal A}}   \eta_{z}^{-1}  \circ_{{\cal A}} Tg  \circ_{{\cal A}} Tf  \circ_{{\cal A}} \eta_{x} \\
 \hspace{210pt} = \eta_{w}^{-1} \circ_{{\cal A}} TSh \circ_{{\cal A}} Tg  \circ_{{\cal A}} Tf  \circ_{{\cal A}} \eta_{x} $ \\
$( h \circ g ) \circ f = ( Sh \circ_{{\cal B}} g ) \circ f = \eta_{w}^{-1} \circ_{{\cal A}} T(Sh \circ_{{\cal B}} g) \circ_{{\cal A}} Tf \circ_{{\cal A}} \eta_{x} \\
\hspace{133pt} = \eta_{w}^{-1} \circ_{{\cal A}} TSh \circ_{{\cal A}} Tg \circ_{{\cal A}} Tf \circ_{{\cal A}} \eta_{x}$
 \item $x \in {\cal A}, y \in {\cal B}, z \in {\cal A}, w \in {\cal B}$, \\
 $h \circ ( g \circ f ) = h \circ ( \eta_{z}^{-1} \circ_{{\cal A}} Tg \circ_{{\cal A}} Tf \circ_{{\cal A}} \eta_{x} )\\
\hspace{52pt} = h \circ_{{\cal B}} S( \eta_{z}^{-1} \circ_{{\cal A}} Tg \circ_{{\cal A}} Tf \circ_{{\cal A}} \eta_{x} )\\
\hspace{52pt} =  h \circ_{{\cal B}} S\eta_{z}^{-1} \circ_{{\cal B}} ST(g \circ_{{\cal B}} f) \circ_{{\cal B}} S\eta_{x} \\
\hspace{52pt} = h \circ_{\cal B} (\epsilon_{Sz} \circ_{{\cal B}} S \eta_{z}) \circ_{{\cal B}}S \eta_{z}^{-1} \circ_{{\cal B}} ST(g \circ_{{\cal B}} f) \circ_{{\cal B}} S \eta_{x} \\
\hspace{52pt} = h \circ_{\cal B} \epsilon_{Sz} \circ_{{\cal B}} ST(g \circ_{{\cal B}} f) \circ_{{\cal B}} S \eta_{x} \\
\hspace{52pt} = h \circ_{\cal B} g \circ_{{\cal B}} f \circ_{{\cal B}} \epsilon_{Sx} \circ_{{\cal B}} S \eta_{x} \\
\hspace{52pt} = h \circ_{\cal B} g \circ_{{\cal B}} f $\\
$( h \circ g ) \circ f = ( h \circ_{{\cal B}} g ) \circ f = ( h \circ_{{\cal B}} g ) \circ_{{\cal B}} f $
 \item $ x \in {\cal A}, y \in {\cal B}, z \in {\cal B}, w \in {\cal A}$, \\
 $h \circ ( g \circ f ) = h \circ ( g \circ_{{\cal B}} f ) = \eta_{w}^{-1} \circ_{{\cal A}} Th \circ_{{\cal A}} T(g \circ_{{\cal B}} f) \circ_{{\cal A}} \eta_{x}$\\
$( h \circ g ) \circ f = ( h \circ_{{\cal B}} g ) \circ f = \eta_{w}^{-1} \circ_{{\cal A}} T(h \circ_{{\cal B}} g) \circ_{{\cal A}} Tf \circ_{{\cal A}} \eta_{x} $
 \item $x \in {\cal A}, y \in {\cal B}, z \in {\cal B}, w \in {\cal B}$, \\
$h \circ ( g \circ f ) = h \circ_{{\cal B}} ( g \circ_{{\cal B}} f ) $\\
$( h \circ g ) \circ f = ( h \circ_{{\cal B}} g ) \circ_{{\cal B}} f $
 \item $x \in {\cal B}, y \in {\cal A}, z \in {\cal A}, w \in {\cal A}$, \\
$h \circ ( g \circ f ) = h \circ (Sg \circ_{{\cal B}} f) = Sh \circ_{{\cal B}} (Sg \circ_{{\cal B}} f)$\\
$( h \circ g ) \circ f = (h \circ_{{\cal A}} g) \circ f = S(h \circ_{{\cal A}} g) \circ_{{\cal B}} f = (Sh \circ_{{\cal B}} Sg) \circ_{{\cal B}} f$
 \item $x \in {\cal B}, y \in {\cal A}, z \in {\cal A}, w \in {\cal B}$, \\
$h \circ ( g \circ f ) = h \circ (Sg \circ_{{\cal B}} f) = h \circ_{{\cal B}} (Sg \circ_{{\cal B}} f)$\\
$( h \circ g ) \circ f = (h \circ_{{\cal B}} Sg) \circ f = (h \circ_{{\cal B}} Sg) \circ_{{\cal B}} f$
 \item $x \in {\cal B}, y \in {\cal A}, z \in {\cal B}, w \in {\cal A}$, \\
$h \circ ( g \circ f ) = h \circ (g \circ_{{\cal B}} f) = h \circ_{{\cal B}} (g \circ_{{\cal B}} f)$\\
$( h \circ g ) \circ f = (\eta_{w}^{-1} \circ_{{\cal A}} Th \circ_{{\cal A}} Tg \circ_{{\cal A}} \eta_{y}) \circ f \\
\hspace{52pt} = S (\eta_{w}^{-1} \circ_{{\cal A}} Th \circ_{{\cal A}} Tg \circ_{{\cal A}} \eta_{y}) \circ_{{\cal B}} f \\
\hspace{52pt} = S\eta_{w}^{-1} \circ_{{\cal B}} ST(h \circ_{{\cal B}} g) \circ_{{\cal B}} S\eta_{y} \circ_{{\cal B}} f \\
\hspace{52pt} = (\epsilon_{Sw} \circ_{{\cal B}} S \eta_{w}) \circ_{{\cal B}}S \eta_{w}^{-1} \circ_{{\cal B}} ST(h \circ_{{\cal B}} g) \circ_{{\cal B}} S \eta_{y} \circ_{\cal B} f \\
\hspace{52pt} = \epsilon_{Sw} \circ_{{\cal B}} ST(h \circ_{{\cal B}} g) \circ_{{\cal B}} S \eta_{y} \circ_{\cal B} f \\
\hspace{52pt} = h \circ_{{\cal B}} g \circ_{{\cal B}} \epsilon_{Sy} \circ_{{\cal B}} S \eta_{y} \circ_{\cal B} f \\
\hspace{52pt} = h \circ_{{\cal B}} g \circ_{\cal B} f$
 \item $x \in {\cal B}, y \in {\cal A}, z \in {\cal B}, w \in {\cal B}$, \\
$h \circ ( g \circ f ) = h \circ_{{\cal B}} ( g \circ_{{\cal B}} f ) $\\
$( h \circ g ) \circ f = ( h \circ_{{\cal B}} g ) \circ_{{\cal B}} f $
 \item $x \in {\cal B}, y \in {\cal B}, z \in {\cal A}, w \in {\cal A}$, \\
$h \circ ( g \circ f ) = h \circ (g \circ_{{\cal B}} f) = Sh \circ_{{\cal B}} (g \circ_{{\cal B}} f)$\\
$( h \circ g ) \circ f = (Sh \circ_{{\cal B}} g) \circ f = (Sh \circ_{{\cal B}} g) \circ_{{\cal B}} f$
 \item $x \in {\cal B}, y \in {\cal B}, z \in {\cal A}, w \in {\cal B}$, \\
$h \circ ( g \circ f ) = h \circ_{{\cal B}} ( g \circ_{{\cal B}} f ) $\\
$( h \circ g ) \circ f = ( h \circ_{{\cal B}} g ) \circ_{{\cal B}} f $
 \item $x \in {\cal B}, y \in {\cal B}, z \in {\cal B}, w \in {\cal A}$, \\
$h \circ ( g \circ f ) = h \circ_{{\cal B}} ( g \circ_{{\cal B}} f ) $\\
$( h \circ g ) \circ f = ( h \circ_{{\cal B}} g ) \circ_{{\cal B}} f $
 \item $x \in {\cal B}, y \in {\cal B}, z \in {\cal B}, w \in {\cal B}$, \\
$h \circ ( g \circ f ) = h \circ_{{\cal B}} ( g \circ_{{\cal B}} f ) $\\
$( h \circ g ) \circ f = ( h \circ_{{\cal B}} g ) \circ_{{\cal B}} f $
 \end{itemize}
 
\item identity law
 \begin{itemize}
 \item $ x \in {\cal A}, y \in {\cal A}$, \\
$f \circ {\rm id}_{x} = f \circ_{{\cal A}} {\rm id}_{x} = f$\\
${\rm id}_{y} \circ f = {\rm id}_{y} \circ_{{\cal A}} f = f$
 \item $x \in {\cal A}, y \in {\cal B}$, \\
$f \circ {\rm id}_{x} = f \circ_{{\cal B}} S{\rm id}_{x} = f \circ_{{\cal B}} {\rm id}_{Sx} = f$\\
${\rm id}_{y} \circ f = {\rm id}_{y} \circ_{{\cal B}} f = f$
 \item $ x \in {\cal B}, y \in {\cal A}$, \\
$f \circ {\rm id}_{x} = f \circ_{{\cal B}} {\rm id}_{x} = f$\\
${\rm id}_{y} \circ f = S{\rm id}_{y} \circ_{{\cal B}} f = {\rm id}_{Sy} \circ_{{\cal B}} f = f$
 \item $x \in {\cal B}, y \in {\cal B}$, \\
$f \circ {\rm id}_{x} = f \circ_{{\cal B}} {\rm id}_{x} = f$\\
${\rm id}_{y} \circ f = {\rm id}_{y} \circ_{{\cal B}} f = f$
 \end{itemize}

\end{itemize}






\begin{definition}
Let $\langle S:{\cal A}\rightarrow{\cal B}, T:{\cal B} \rightarrow {\cal A},  \eta :I_{{\cal A}} \rightarrow TS , \epsilon :ST \rightarrow I_{{\cal B}} \rangle$ be an adjoint equivalence, let ${\cal A}\sqcup \hspace{-.76em} \mid {\cal B}$ be the equivalence fusion. We define the projections $u,v$ as follows:

\begin{list}{$\bullet$}{}
\item $u:{\cal A}\sqcup \hspace{-.76em} \mid {\cal B} \longrightarrow A$ \\
 object-function 
$ u:{\bf Ob}{\rm (}{\cal A}\sqcup \hspace{-.76em} \mid {\cal B}{\rm )} \longrightarrow {\bf Ob}{\rm (}{\cal A}{\rm )} $ \\
$ \hspace{141pt} x  \longmapsto ux := \left\{ \begin{array}{ll}
x & {\rm (}x \in {\cal A}{\rm )} \\
Tx & {\rm (}x \in {\cal B}{\rm )} \\
\end{array} \right. $ \\
 hom-functions $u:{\bf Hom}{\rm (}x,y{\rm )} \longrightarrow {\cal A}{\rm (}ux,uy{\rm )}$ \\
\hspace{106pt} $\langle f,x,y \rangle \longmapsto uf := \left\{ \begin{array}{ll} 
f & {\rm (}x,y \in {\cal A}{\rm )} \\
Tf & {\rm (}x,y \in {\cal B}{\rm )} \\
Tf \circ_{{\cal A}} \eta_{x} & {\rm (}x \in {\cal A}, y \in {\cal B}{\rm )} \\
\eta_{y}^{-1} \circ_{{\cal A}} Tf & {\rm (}x \in {\cal B}, y \in {\cal A}{\rm )} \\
\end{array} \right. $ \\

\item $v:{\cal A}\sqcup \hspace{-.76em} \mid {\cal B} \longrightarrow B$ \\
 object-function $v:{\bf Ob}{\rm (}{\cal A}\sqcup \hspace{-.76em} \mid {\cal B}{\rm )} \longrightarrow {\bf Ob}{\rm (}{\cal B}{\rm )}$ \\
\hspace{141pt} $x \longmapsto vx := \left\{ \begin{array}{ll}
Sx & {\rm (}x \in {\cal A}{\rm )} \\
x & {\rm (}x \in {\cal B}{\rm )} \\
\end{array} \right. $ \\
 hom-functions $v:{\bf Hom}{\rm (}x,y{\rm )} \longrightarrow {\cal B}{\rm (}ux,uy{\rm )}$�@\\
\hspace{106pt} $\langle f,x,y \rangle \longmapsto vf := \left\{ \begin{array}{ll}
Sf & {\rm (}x,y \in {\cal A}{\rm )} \\
f & {\rm ( others )} \\ 
\end{array} \right.$ 
\end{list}
\end{definition}

\begin{proposition}
The projections $u,v$ are functors.
\end{proposition}

\pf We show that $u,v$ preserve composition of morphisms and identity morphism by case analysis.
\begin{itemize}
\item $u$ preserves composition of morphisms
 \begin{itemize}
 \item $x \in {\cal A}, y \in {\cal A}, z \in {\cal A}$, \\
 $u(g \circ f) = u(g \circ_{{\cal A}} f) = g \circ_{{\cal A}} f$ \\
$ug \circ_{{\cal A}} uf = g \circ_{{\cal A}} f$ 
 \item $x \in {\cal A}, y \in {\cal A}, z \in {\cal B}$, \\
 $u(g \circ f) = u(g \circ_{{\cal B}} Sf) = T(g \circ_{{\cal B}} Sf) \circ_{{\cal A}} \eta_{x}=Tg \circ_{{\cal A}} TSf \circ_{{\cal A}} \eta_{x}$ \\
$ug \circ_{{\cal A}} uf =(Tg \circ_{{\cal A}} \eta_{y}) \circ_{{\cal A}} f = Tg \circ_{\cal A} TSf \circ_{{\cal A}} \eta_{x}$ 
 \item $x \in {\cal A}, y \in {\cal B}, z \in {\cal A}$, \\
 $u(g \circ f) =u(\eta_{z}^{-1} \circ_{{\cal A}} Tg \circ_{{\cal A}} Tf \circ_{{\cal A}} \eta_{x})=\eta_{z}^{-1} \circ_{{\cal A}} Tg \circ_{{\cal A}} Tf \circ_{{\cal A}} \eta_{x}$ \\
$ug \circ_{{\cal A}} uf=(\eta_{z}^{-1} \circ_{{\cal A}} Tg) \circ_{{\cal A}} (Tf \circ_{{\cal A}} \eta_{x})$ 
 \item $x \in {\cal A}, y \in {\cal B}, z \in {\cal B}$, \\
 $u(g \circ f) = u(g \circ_{{\cal B}} f)=T(g \circ_{{\cal B}} f) \circ_{{\cal A}} \eta_{x}=Tg \circ_{{\cal A}} Tf \circ_{{\cal A}} \eta_{x}$ \\
$ug \circ_{{\cal A}} uf = Tg \circ_{{\cal A}} (Tf \circ_{{\cal A}} \eta_{x})$
 \item $x \in {\cal B}, y \in {\cal A}, z \in {\cal A}$, \\
$u(g \circ f) = u(Sg \circ_{{\cal B}} f) = \eta_{z}^{-1} \circ_{{\cal A}} T(Sg \circ_{{\cal B}} f) = \eta_{z}^{-1} \circ_{{\cal A}} TSg \circ_{{\cal B}} Tf$ \\
$ug \circ_{{\cal A}} uf = g \circ_{{\cal A}} (\eta_{y}^{-1} \circ_{{\cal A}} Tf) = \eta_{z}^{-1} \circ_{{\cal A}} TSg \circ_{\cal A} Tf$
 \item $ x \in {\cal B}, y \in {\cal A}, z \in {\cal B}$, \\
 $u(g \circ f) = u(g \circ_{{\cal B}} f) = T(g \circ_{{\cal B}} f) = Tg \circ_{{\cal A}} Tf$ \\
$ug \circ_{{\cal A}} uf = (Tg \circ_{{\cal A}} \eta_{y}) \circ_{{\cal A}} (\eta_{y}^{-1} \circ_{{\cal A}} Tf)= Tg \circ_{{\cal A}} Tf$
 \item $ x \in {\cal B}, y \in {\cal B}, z \in {\cal A}$, \\
$u(g \circ f) = u(g \circ_{{\cal B}} f) = \eta_{z}^{-1} \circ_{{\cal A}} T(g \circ_{{\cal B}} f) = \eta_{z}^{-1} \circ_{{\cal A}} Tg \circ_{{\cal A}} Tf$ \\
$ug \circ_{{\cal A}} uf = (\eta_{z}^{-1} \circ_{{\cal A}} Tg) \circ_{{\cal A}} Tf$
 \item $x \in {\cal B}, y \in {\cal B}, z \in {\cal B}$, \\
 $u(g \circ f) = u( g \circ_{{\cal B}} f ) = T(g \circ_{{\cal B}} f)=Tg \circ_{{\cal A}} Tf$ \\
$ug \circ_{{\cal A}} uf =Tg \circ_{{\cal A}} Tf$
 \end{itemize}
 
\item $u$ preserves identity morphisms
 \begin{itemize}
 \item $x \in {\cal A}$, \\
$u({\rm id}_{x})={\rm id}_{x}={\rm id}_{ux}$
 \item $x \in {\cal B}$, \\
$u({\rm id}_{x})=T{\rm id}_{x}={\rm id}_{Tx}={\rm id}_{ux}$
 \end{itemize}
 
 \item $v$ preserves composition of morphisms
 \begin{itemize}
 \item $x \in {\cal A}, y \in {\cal A}, z \in {\cal A}$, \\
$v(g \circ f) =v(g \circ_{{\cal A}} f)=S(g \circ_{{\cal A}} f) = Sg \circ_{{\cal B}} Sf$ \\
$vg \circ_{{\cal B}} vf =Sg \circ_{{\cal A}} Sf$
 \item $x \in {\cal A}, y \in {\cal A}, z \in {\cal B}$, \\
$v(g \circ f) = v(g \circ_{{\cal B}} Sf) = g \circ_{{\cal B}} Sf$ \\
$vg \circ_{{\cal B}} vf = g \circ_{{\cal B}} Sf$
 \item $ x \in {\cal A}, y \in {\cal B}, z \in {\cal A}$, \\
$v(g \circ f) = v(\eta_{z}^{-1} \circ_{{\cal A}} Tg \circ_{{\cal A}} Tf \circ_{{\cal A}} \eta_{x}) \\
\hspace{40pt} =S(\eta_{z}^{-1} \circ_{{\cal A}} Tg \circ_{{\cal A}} Tf \circ_{{\cal A}} \eta_{x}) \\
\hspace{40pt} = S\eta_{z}^{-1} \circ_{{\cal B}} ST(g \circ_{{\cal B}} f) \circ_{{\cal B}} S\eta_{x} \\
\hspace{40pt} = g \circ_{{\cal B}} f $ \\
$vg \circ_{{\cal B}} vf = g \circ_{{\cal B}} f$
\item $ x \in {\cal A}, y \in {\cal B}, z \in {\cal B}$, \\
$v(g \circ f) = v(g \circ_{{\cal B}} f) = g \circ_{{\cal B}} f$ \\
$vg \circ_{{\cal B}} vf = g \circ_{{\cal B}} f$
\item $x \in {\cal B}, y \in {\cal A}, z \in {\cal A}$, \\
$v(g \circ f) = v(Sg \circ_{{\cal B}} f) = Sg \circ_{{\cal B}} f$ \\
$vg \circ_{{\cal B}} vf = Sg \circ_{{\cal B}} f$
\item $x \in {\cal B}, y \in {\cal A}, z \in {\cal B}$, \\
$v(g \circ f) = v(g \circ_{{\cal B}} f) = g \circ_{{\cal B}} f$ \\
$vg \circ_{{\cal B}} vf = g \circ_{{\cal B}} f$
\item $x \in {\cal B}, y \in {\cal B}, z \in {\cal A}$, \\
$v(g \circ f) = v(g \circ_{{\cal B}} f) = g \circ_{{\cal B}} f$ \\
$vg \circ_{{\cal B}} vf = g \circ_{{\cal B}} f$ 
\item $ x \in {\cal B}, y \in {\cal B}, z \in {\cal B}$, \\
$v(g \circ f) = v(g \circ_{{\cal B}} f) = g \circ_{{\cal B}} f$ \\
$vg \circ_{{\cal B}} vf = g \circ_{{\cal B}} f$
 \end{itemize}
 
\item $v$ preserves identity morphisms
 \begin{itemize}
 \item $ x \in {\cal A}$ \\
$v({\rm id}_{x})=S{\rm id}_{x}={\rm id}_{Sx}={\rm id}_{vx}$
 \item $x \in {\cal B}$ \\
$v({\rm id}_{x})={\rm id}_{x}~{\rm id}_{vx}$
 \end{itemize}
 
\end{itemize}



\begin{proposition}
The projections $u,v$ are surjective on objects, full and faithful.
\end{proposition}

\pf It's trivial by definitions that $u,v$ are surjective on objects. So we check fullness and faithfulness.
\begin{itemize}
 \item $u$ is full and faithful
 \begin{itemize}
 \item $x,y \in {\cal A}$, \\
 $u:{\bf Hom}(x,y)=\{ \langle f,x,y \rangle \mid f \in {\cal A}(x,y) \} \ni \langle f,x,y \rangle \mapsto f \in {\cal A}(x,y)$ is bijective.
 \item $x,y \in {\cal B}$, \\
 $T:{\cal B}{\rm (}x,y{\rm )} \rightarrow {\cal A}{\rm (}Tx,Ty{\rm )}$ is bijective. Therefore $u: {\bf Hom}{\rm (}x,y{\rm )}=\{ \langle f,x,y \rangle \mid f \in {\cal B}(x,y) \} \ni \langle f,x,y \rangle \mapsto f \in {\cal A}(x,y) \ni \langle f,x,y \rangle \mapsto Tf \in {\cal A}{\rm (}Tx,Ty{\rm )}={\cal A}{\rm (}ux,uy{\rm )}$ is bijective.
 \item $x \in {\cal A}, y \in {\cal B}$, \\
${\cal B}{\rm (}Sx,y{\rm )} \ni f \mapsto Tf \circ_{{\cal A}} \eta_{x} \in {\cal A}{\rm (}x,Ty{\rm )} $ is the right adjunct of each $f$, and bijective. Therefore $u: {\bf Hom}{\rm (}x,y{\rm )}=\{ \langle f,x,y \rangle \mid f \in {\cal B}(Sx,y) \} \ni \langle f,x,y \rangle \mapsto Tf \circ_{{\cal A}} \eta_{x} \in {\cal A}{\rm (}x,Ty{\rm )}={\cal A}{\rm (}ux,uy{\rm )}$ is bijective.
 \item $x \in {\cal B} , y \in {\cal A}$, \\
${\cal B}{\rm (}x,Sy{\rm )} \ni f \mapsto \eta_{y}^{-1} \circ_{{\cal A}} Tf \in {\cal A}{\rm (}Tx,y{\rm )}$ is the left adjunct of each $f$, and bijective. Therefore $u: {\bf Hom}{\rm (}x,y{\rm )}=\{ \langle f,x,y \rangle \mid f \in {\cal B}(x,Sy) \} \ni \langle f,x,y \rangle \mapsto \eta_{y}^{-1} \circ_{{\cal A}} Tf \in {\cal A}{\rm (}Tx,y{\rm )}={\cal A}{\rm (}ux,uy{\rm )}$
 \end{itemize}
 
 \item $v$ is full and faithful
 \begin{itemize}
 \item $x,y \in {\cal A}$, \\
$S:{\cal A}{\rm (}x,y{\rm )} \rightarrow {\cal B}{\rm (}Sx,Sy{\rm )}$ is bijective. Therefore $v:{\bf Hom}{\rm (}x,y{\rm )}=\{ \langle f,x,y \rangle \mid f \in {\cal A}(x,y) \} \ni \langle f,x,y \rangle \mapsto Sf \in {\cal B}{\rm (}Sx,Sy{\rm )}={\cal B}{\rm (}vx,vy{\rm )}$ is bijective.
 \item $x,y \in {\cal B}$, \\
$v:{\bf Hom}{\rm (}x,y{\rm )}=\{ \langle f,x,y \rangle \mid f \in {\cal B}(x,y) \} \ni \langle f,x,y \rangle \mapsto f \in {\cal B}{\rm (}x,y{\rm )}={\cal B}{\rm (}vx,vy{\rm )}$ is bijective.
 \item $x \in {\cal A}, y \in {\cal B}$, \\
$v:{\bf Hom}{\rm (}x,y{\rm )}=\{ \langle f,x,y \rangle \mid f \in {\cal B}(Sx,y) \} \ni \langle f,x,y \rangle \mapsto f \in {\cal B}{\rm (}Sx,y{\rm )}={\cal B}{\rm (}vx,vy{\rm )}$ is bijective.
 \item $x \in {\cal B} , y \in {\cal A}$, \\
$v:{\bf Hom}{\rm (}x,y{\rm )}=\{ \langle f,x,y \rangle \mid f \in {\cal B}(x,Sy) \} \ni \langle f,x,y \rangle \mapsto f \in {\cal B}{\rm (}x,Sy{\rm )}={\cal B}{\rm (}vx,vy{\rm )}$ is bijective.
 \end{itemize}
 
\end{itemize}


\begin{thm}
Let ${\cal A}$ and ${\cal B}$ be categories. ${\cal A}$ is ordinary equivalent to ${\cal B}$ 
if and only if ${\cal A}$ is span equivalent to ${\cal B}$. 
\end{thm}

\pf Let ${\cal A}$ be  ordinary equivalent to ${\cal B}$, then ${\cal A}$ is adjoint equivalent to ${\cal B}$. Thus there exists a adjoint equivalence between ${\cal A}$ and ${\cal B}$. So we can construct the equivalence fusion and the projections. By Propositions, they are span equivalence. Therefore ${\cal A}$ is span equivalent to ${\cal B}$. \\
On the other hand, let ${\cal A}$ be span equivalent to ${\cal B}$. Then there exists a span equivalence $\langle {\cal C},u,v \rangle$ between ${\cal A}$ and ${\cal B}$, and ${\cal C}$ is ordinary equivalent to both ${\cal A}$ and ${\cal B}$. Therefore ${\cal A}$ is ordinary equivalent to ${\cal B}$.

\begin{rem}
Let ${\cal A}$ be presheaf category. The forgetful functor
\[
U: {\cal A} \mathchar`- {\bf Cat} \longrightarrow {\cal A} \mathchar`- {\bf Gph}
\]
is monadic. (Proposition F 1.1 in [Leinster 2004])
\end{rem}

Let ${\cal A}={\bf Set}$, the induced monad $T_1$ is the free strict $1$-category monad, ${\bf Set} \mathchar`- {\bf Grp}$ is the category of $1$-globular sets. Hence 
\[
T_1 \mathchar`- {\bf Alg} \cong {\bf Cat}
\]
Moreover, the category of weak $1$-category ${\bf Wk} \mathchar`- 1 \mathchar`- {\bf Cat}$ is isomorphic to $T_1 \mathchar`- {\bf Alg}$. Therefore
\[
{\bf Wk} \mathchar`- 1 \mathchar`- {\bf Cat} \cong T_1 \mathchar`- {\bf Alg} \cong {\bf Cat}
\]


\begin{proposition}
Let $F: {\bf Cat} \rightarrow {\bf Wk \mathchar`- 1 \mathchar`- Cat}$ be the isomorphism above. let ${\cal A}$ and ${\cal B}$ be categories. ${\cal A}$ is span equivalent to ${\cal B}$ in ${\bf Cat}$ if and only if $F({\cal A})$ is span equivalent to $F({\cal B})$ in ${\bf Wk \mathchar`- 1 \mathchar`- Cat}$.
\end{proposition}

\pf

\begin{thm} 
Let ${\cal A}$ and ${\cal B}$ be categories. ${\cal A}$ is ordinary equivalent to ${\cal B}$ 
in ${\bf Cat}$ if and only if F$({\cal A})$ is span equivalent to F$({\cal B})$ in ${\bf Wk \mathchar`- 1 \mathchar`- Cat}$. 
\end{thm}


\newpage



\begin{refs}

\bibitem [Lamport, 1986]{LUG} L. Lamport, Latex User's Guide \&
Reference Manual. Addison-Wesley (fifth edition), 1986.

\end{refs}



\end{document}
